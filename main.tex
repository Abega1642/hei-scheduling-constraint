\documentclass[12pt,a4paper]{article}
\usepackage[utf8]{inputenc}
\usepackage[english]{babel}
\usepackage{amsmath}
\usepackage{amsfonts}
\usepackage{amssymb}
\newtheorem{theorem}{Constraint}
\author{Abegà Razafindratelo}

\title{HEI Scheduling constraints problems}
\begin{document}
\maketitle
\section*{Notations}
\begin{itemize}
	\item Let $p \in \{-1;1\} $ be a an indicator constant.
	$$
	p = 
	\begin{cases}
		\hspace{0.3cm} 1 \text{ , \quad if the course is presential} \\
		-1 \text{ , \quad otherwise}
	\end{cases} 
	$$
	\item $G$ denotes the set of all group $g$.
	
	\item Let $I \subset \mathbb{N}$ such as $I = \lbrace 1, 2, ..., n \rbrace$. Where $\#I = n = \#AC$
	
	\item All element of the set of all awarded courses $AC$ will be indexed. In other words :
	$$
	 ac_{i} \in AC,\ i \in I
	$$
	
	\item Let $t_{ac_{i}}$ and $c_{ac_{i}}$ be the corresponding teacher and the corresponding course resp. to  $ac_{i}$.
	
	\item Let $J \subset I$ such as :
	$$
	\forall \ i, j \in J \ : \left(t_{ac_{i}} = t_{ac_{j}} \right) \text{ and } 
	\left(c_{ac_{i}} = c_{ac_{j}} \right)
	$$ 
	
	\item Let $K \subset I$ such as :
	$$
	\forall \ i, j \in K, \ i \neq j : \left(t_{ac_{i}} = t_{ac_{j}} \right) \text{ and } 
	\left(c_{ac_{i}} \neq c_{ac_{j}} \right)
	$$ 
	
	\item Let $AC_{g} \subset AC$, the set of all awarded course related to a group $g$.
	

\end{itemize}

\section*{About the Constraints}

\begin{theorem}
A teacher can't teach two or more different groups at the same time for a same course if it is face-to-face session but can if it is a video conference session.
\end{theorem}
$
\forall \ s \in S,\ \forall \ d \in D :
$
\begin{equation}
2p \left[ \sum_{r\in R}\left(\sum_{ac_{j} \in AC, j \in J}{o_{ac_{j}, d, r, s}}\right)\right] \leq 3p - 1
\end{equation}
\\ \\
\textbf{\textit{\underline{Otherwise :}}}
if $p$ is the binary variable such as :
\[
	p = 
	\begin{cases}
		\hspace{0.3cm} 1 \text{ , \quad if the course is presential} \\
		\hspace{0.3cm}0 \text{ , \quad otherwise}
	\end{cases} 
\]	
let $M \in \mathbb{N}$ be a majoration constant. \\ \\
Therefore :
\begin{equation}
\sum_{r\in R}\left(\sum_{ac_{j} \in AC, j \in J}{o_{ac_{j}, d, r, s}}\right) \leq 1 - M(p - 1)
\end{equation}

\begin{theorem}
A teacher cannot simultaneously teach two or more different courses.
\end{theorem}
$
\forall \ s \in S,\ \forall \ d \in D : 
$
\begin{equation}
\sum_{r\in R}\left(\sum_{ac_k \in AC, k \in K}{o_{ac_{k}, d, r, s}}\right) \leq 1
\end{equation}

\begin{theorem}
A group cannot simultaneously have two or more different course sessions.
\end{theorem}
$
\forall \ g \in G, \ \forall	\ s \in S, \ \forall \ d \in D :
$
\begin{equation}
\sum_{r\in R}\left(\sum_{ac_i \in AC_g}{o_{ac_{i}, d, r, s}}\right) \leq 1
\end{equation}
\begin{theorem}
A group should have 2 hours of break per day.
\end{theorem}
$
\forall \ g \in G, \ \forall \ d \in D :
$
\begin{equation}
\sum_{ac_{i} \in AC_{g}}\left(\sum_{r\in R}{\sum_{s \in S}{o_{ac_{i}, d, r, s}}}\right) \leq 3
\end{equation}
\begin{theorem}
Only one session per course in a day.
\end{theorem}
$
\forall \ ac_i \in AC,\ i \in I, \ \forall \ d \in D :
$
\begin{equation}
\left(\sum_{r\in R}{\sum_{s \in S}{o_{ac_{i}, d, r, s}}}\right) \leq 1
\end{equation}
\begin{theorem}
Give one day without the same course after a session of this course.
\end{theorem}
$
\forall \ d \in D, \ \forall ac_i \in AC, \ i \in I:
$
\begin{equation}
\left(\sum_{r\in R}{\sum_{s \in S}{o_{ac_{i}, d + 1_d, r, s}}}\right) + \left(\sum_{r\in R}{\sum_{s \in S}{o_{ac_{i}, d, r, s}}}\right) \leq 1
\end{equation}
where $1_d$ is the unit of day.

\begin{theorem}
Suitable room for every course session.
\end{theorem}

\begin{itemize}
	\item Let $g_{ac_{i}}$ be the corresponding group to a ${ac_{i}}$.
	\item Let $\#g_{ac_{i}}$ be the group size of a group $g_{ac_{i}}$.
	\item Let $rc_r$ be the room capacity of a room $r$.
\end{itemize}

$
\forall ac_i \in AC, \ i \in I, \forall \ r \in R, \ \forall \ s \in S, \forall \ d \in D 
$
\begin{equation}
rc_r - \#g_{ac_{i}} \geq 0
\end{equation}

\begin{theorem}
Finish the total hour of every course.
\end{theorem}
Let $D(ac_i, d_q)$ be the total duration of a $ac_i$ within a time interval $\left[d_0, d_q\right] \subset D, \ d_0 \leq d_q$. \\ \\
$
\forall \ ac_{i} \in AC, \ i \in I :
$ \\
\begin{equation}
\sum_{d = d_0 }^{d_q}\sum_{r \in R}\sum_{s \in S} {o_{ac_{i}, d, r, s}} = D(ac_{i})
\end{equation}


\end{document}

