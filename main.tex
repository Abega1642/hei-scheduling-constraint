\documentclass[12pt,a4paper]{article}
\usepackage[utf8]{inputenc}
\usepackage[english]{babel}
\usepackage{amsmath}
\usepackage{amsfonts}
\usepackage{amssymb}
\usepackage{amsthm}
\newtheorem{theorem}{Constraint}

\author{Abegà Razafindratelo}
\title{HEI Scheduling constraints problems}

\begin{document}
\maketitle
\section*{Notations}
\begin{itemize}
	
	\item $G$ denotes the set of all group $g$.
	
	\item $T$ denotes the set of all teacher $t$.
	
	\item $C$ denotes the set of all course $c$.
	
	\item $R$ the set of all room $r$.
	
	\item All element of the set of all awarded courses $AC$ will be indexed. In other words :
	$$
	 ac_{i} \in AC,\ i \in \mathbb{N}
	$$
	
	\item Let $t(ac_{i})$ and $c(ac_{i})$ be the corresponding teacher and the corresponding course resp. to  $ac_{i}$.
	
	\item Let $STC_{t,c} \subset AC$  be the set of all awarded course which has same teacher and course (Same Teacher and Course):
	$$
	\forall \ i, j \in \mathbb{N} \ : \ \left(ac_{i}, ac_{j} \in  STC_{t,c}\right)                            \Longleftrightarrow \left[t(ac_{i}) = t(ac_{j})  = t \right] \text{ and } 
	\left[c(ac_{i}) = c(ac_{j})  =  c \right]
	$$ 
	\underline{Example} : \\
	
	$t = DrToky$, $c = PROG1$, and $G = \lbrace K1, K2, K3 \rbrace$. So, we can have \\
	$
		STC_{DrToky, PROG1} = \lbrace a_1, a_2, a_3 \rbrace	\\
	$ 
	\[
	\text{where :}
		\begin{cases}
			ac_1 = \left(PROG1, DrToky, K1\right) \\
			ac_2 = \left(PROG1, DrToky, K2\right) \\
			ac_3 = \left(PROG1, DrToky, K3\right)
		\end{cases}
	\]
	
	\item Let $STDC_{t} \subset AC$ be the set of all awarded course which has same teacher but different course (Same Teacher but Different Course) :
	$$
	\forall \ i, j \in \mathbb{N} \ : \ \left(ac_{i}, ac_{j} \in  STDC_{t}\right)                            \Longleftrightarrow \left[t(ac_{i}) = t(ac_{j})  = t \right] \text{ and } 
	\left[c(ac_{i}) \neq c(ac_{j}) \right]
	$$ 
	\underline{Example} : \\
	$t = DrToky$, and $G = \lbrace K1, K2, K3 \rbrace$. So, we can have \\
	$
		STDC_{DrToky} = \lbrace a_1, a_2, a_3, a_4 \rbrace	\\
	$ 
	\[
	\text{where :}
		\begin{cases}
			ac_1 = \left(PROG1, Dr Toky, K1\right) \\
			ac_2 = \left(WEB1, Dr Toky, K1\right) \\
			ac_3 = \left(DONNEES1, DrToky, K2\right)\\
			ac_4 = \left(PROG2, DrToky, K3\right)
		\end{cases}
	\]
	
	\item Let $AC_{g} \subset AC$, the set of all awarded course related to a group $g$. Which means that only $ac_i$ that $g$ as its group will be in the set $AC_{g}$.

\end{itemize}
\newpage
\section*{About the Constraints}

\begin{theorem}
A teacher can't teach two or more different groups at the same time for a same course if it is face-to-face session but can if it is a video conference session.
\end{theorem}


\vspace{0.5cm}Let $pres(d, STC_{t,c})$ be the binary variable such as :
\[
	pres(d, STC_{t,c}) = 
	\begin{cases}
		\hspace{0.3cm} 1 \text{ , if the course related to $STC_{t,c}$ is presential at a given $d$ day} \\
		\hspace{0.3cm}0 \text{ , otherwise}
	\end{cases} 
\]	

Let $M \in \mathbb{N^{*}}$ be a constant limiting the number of allowed courses. \\ \\
Therefore : \\

$
\forall \ s \in S,\ \forall \ d \in D, \forall \ t \in T, \ \forall \ c \in C :
$
\begin{equation}
\sum_{r\in R}\left(\sum_{ac_{i} \in STC_{t,c}}{o_{ac_{i}, d, r, s}}\right) \leq 1 + M(1 - pres(d, STC_{t,c}))
\end{equation}

\vspace{1cm}\textbf{\underline{Example:}}\vspace{0.3cm} \\ 
Let's take an example for course $c = PROG1$ and teacher $t = Dr Toky$ and all groups $K1$, $K2$, $K3$, $K4$, $K5$. \\ \\
We define the set $ac_i$ as follow :
$$
\begin{cases}
	ac_1 = \left(PROG1, Dr Toky, K1\right) \\
	ac_2 = \left(PROG1, Dr Toky, K2\right) \\
	ac_3 = \left(PROG1, Dr Toky, K3\right) \\
	ac_4 = \left(PROG1, Dr Toky, K4\right) \\
	ac_5 = \left(PROG1, Dr Toky, K5\right) \\
\end{cases}
$$

\vspace{0.5cm}Thus, the set $STC_{PROG1, DrToky}$ is :
\[
STC_{PROG1, DrToky} = \lbrace ac_1, ac_2, ac_3, ac_4, ac_5 \rbrace
\]
Which means here that all elements of this $STC_{PROG1, DrToky}$ has the same teacher and same course.\\
Now, if we say that for a $d$ day, the $PROG1$ course will be presential, we will have : \\ 

\[
pres(d, STC_{t,c}) = 1 \quad \Longrightarrow \quad 
\sum_{r\in R}\left(\sum_{ac_{i} \in STC_{t,c}}{o_{ac_{i}, d, r, s}}\right) \leq 1 + M(1 - 1)
\]

\[
\Longleftrightarrow \quad
\sum_{r\in R}\left(\sum_{ac_{i} \in STC_{t,c}}{o_{ac_{i}, d, r, s}}\right) \leq 1
\]
Which tells us that only one session of that course is allowed at that given time slot.\\

\vspace{0.5cm} Now, if the course is online : \\
\[
pres(d, STC_{t,c}) = 0 \quad \Longrightarrow \quad 
\sum_{r\in R}\left(\sum_{ac_{i} \in STC_{t,c}}{o_{ac_{i}, d, r, s}}\right) \leq 1 + M(1 - 0)
\]

\[
\Longleftrightarrow \quad
\sum_{r\in R}\left(\sum_{ac_{i} \in STC_{t,c}}{o_{ac_{i}, d, r, s}}\right) \leq 1 + M
\]

which means that multiple session of that course (depending on $M$) is allowed\vspace{1cm}



\begin{theorem}
A teacher cannot simultaneously teach two or more different courses.
\end{theorem}
$
\forall \ s \in S,\ \forall \ d \in D, \forall \ t \in T:
$
\begin{equation}
\sum_{r\in R}\left(\sum_{ac_i \in STDC_{t}}{o_{ac_{i}, d, r, s}}\right) \leq 1
\end{equation}

\vspace{1cm}\textbf{\underline{Example:}}\vspace{0.3cm} \\ 
Let's take an example for courses $c_1 = PROG1$, $c_2 = WEB1$ and teacher $t = Dr Toky$ and all groups $K1$, $K2$. \\ \\
Now, the statement is that $DrToky$ cannot teach $WEB1$ and $PROG1$ at the same time. So, we have to make sure that \textbf{event if we have different courses}, $DrToky$ don't teach both of those courses at a given slot $s$.\\ 
\[
\text{Let : }
\begin{cases}
	ac_1 = (PROG1, DrToky, K1) \\
	ac_2 = (WEB1, DrToky, K2)
\end{cases}
\]
It is clear that $ac_1$, $ac_2$ $\in STDC_{DrToky}$ and we can only allow session for $ac_1$ \textbf{or} for $ac_2$. \\
It is abvious that if the awarded courses don't have the same teacher but have the same course, we can allow both of them \vspace{1cm}.



\begin{theorem}
Only one session is allowed in a room at a given time slot.
\end{theorem}
$
\forall \ r \in R, \ \forall	\ s \in S, \ \forall \ d \in D :
$
\begin{equation}
\sum_{ac_i \in AC}{o_{ac_{i}, d, r, s}} \leq 1
\end{equation}

\begin{theorem}
A group cannot simultaneously have two or more different course sessions.
\end{theorem}
$
\forall \ g \in G, \ \forall	\ s \in S, \ \forall \ d \in D :
$
\begin{equation}
\sum_{r\in R}\left(\sum_{ac_i \in AC_g}{o_{ac_{i}, d, r, s}}\right) \leq 1
\end{equation}

\vspace{1cm}\textbf{\underline{Example:}}\vspace{0.3cm} \\ 
Let's take for example the group $J1$. This group has for example the $MGT2$, $PROG3$ courses.

\[
	\text{Let : }
	\begin{cases}
	ac_1 = (MGT2, DrLou, J1) \\
	ac_2 = (PROG3, MrRyan, J1)
	\end{cases}
\]

It is abvious that this doesn't concern any other group and we can't allow $J1$ to have $ac_1$ and $ac_2$ at the same time.\vspace{1cm}



\begin{theorem}
A group should have 2 hours of break per day.
\end{theorem}
$
\forall \ g \in G, \ \forall \ d \in D :
$
\begin{equation}
\sum_{ac_{i} \in AC_{g}}\left(\sum_{r\in R}{\sum_{s \in S}{o_{ac_{i}, d, r, s}}}\right) \leq 3
\end{equation}

It is clear that this does not concern at all the other groups apart from the given group $g$.\vspace{1cm}



\begin{theorem}
Only one session per course in a day.
\end{theorem}
$
\forall \ ac_i \in AC, \ \forall \ d \in D :
$
\begin{equation}
\left(\sum_{r\in R}{\sum_{s \in S}{o_{ac_{i}, d, r, s}}}\right) \leq 1
\end{equation}

\vspace{1cm}
\begin{theorem}
Give one day without the same course after a session of this course.
\end{theorem}
$
\forall \ d \in D, \ \forall ac_i \in AC :
$
\begin{equation}
\left(\sum_{r\in R}{\sum_{s \in S}{o_{ac_{i}, d + 1_d, r, s}}}\right) + \left(\sum_{r\in R}{\sum_{s \in S}{o_{ac_{i}, d, r, s}}}\right) \leq 1
\end{equation}
where $1_d$ is the unit of day. So, $d + 1_d$ means the next day of $d$\\

\vspace{0.25cm}This inequality states that the number of occurences of an awarded course in day $d$ and the following day $d + 1_d$ should be at most one. Which means that whether the awarded course is scheduled the $d$ day or it is scheduled in the next day.
\vspace{1cm}



\begin{theorem}
Suitable room for every course session.
\end{theorem}

\begin{itemize}
	\item Let $g_{ac_{i}}$ be the corresponding group to a ${ac_{i}}$.
	\item Let $size(g_{ac_{i}})$ be the group size of a group $g_{ac_{i}}$.
	\item Let $capacity(r)$ be the room capacity of a room $r$.
\end{itemize}

$
\forall ac_i \in AC, \forall \ r \in R, \ \forall \ s \in S, \forall \ d \in D 
$
\begin{equation}
o_{ac_i, d, r, s} \cdot capacity(r) \geq  size(g_{ac_{i}})
\end{equation} \vspace{0.6cm}

\begin{theorem}
Finish the total hour of every course.
\end{theorem}
Let $D(ac_i,d_{start}, d_{end})$ be the needed total duration in hour of a $ac_i$ within a time interval $\left[d_{start}, d_{end}\right] \subset D, \ d_{start} \leq d_{end}$. \\ \\
$
\forall \ ac_{i} \in AC:
$ \\
\begin{equation}
2 \cdot \left(\sum_{d = d_{start} }^{d_{end}}\sum_{r \in R}\sum_{s \in S} {o_{ac_{i}, d, r, s}}\right) = D(ac_{i}, d_{start}, d_{end})
\end{equation}

\vspace{1cm}\textbf{\underline{Example:}}\vspace{0.3cm} \\ 
$ac_1 = (PROG1, DrToky, K1)$, $d_{start}$ = 10 march 2025 and $d_{end}$ = 17 march 2025. \\

If we need to have 10 hours of $ac_1$, we have then : $D(ac_1,d_{start}, d_{end}) = 10$. \\

Therefore, we should have :

\[
2 \cdot \left(\sum_{d = d_{start} }^{d_{end}}\sum_{r \in R}\sum_{s \in S} {o_{ac_{1}, d, r, s}}\right) = 10)
\]

\[
\Longleftrightarrow \quad 
\sum_{d = d_{start} }^{d_{end}}\sum_{r \in R}\sum_{s \in S} {o_{ac_{1}, d, r, s}} = 5)
\]
Which gives us that we should have 5 session of $ac_1$ during the interval of 10 march and 17 march to get those required 10 hours since $o_{ac_{1}, d, r, s}$ gives us 2 hours of sessions.


\end{document}

